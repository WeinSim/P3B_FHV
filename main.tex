\documentclass{article}

\usepackage[utf8]{inputenc}
\usepackage[german]{babel}
\usepackage[
    a4paper, top=2.5cm, bottom=2.5cm, left=2.5cm, right=2.5cm, marginparwidth=1.75cm
]{geometry}
\usepackage{amsmath}
\usepackage{amsfonts}
\usepackage{enumitem}
\usepackage[
    colorlinks=true, 
    citecolor=black,
    filecolor=black,
    linkcolor=black,
    urlcolor=blue
]{hyperref}
\usepackage{graphicx}
\usepackage{amssymb}
\usepackage{float}
\usepackage{pdfpages}
\usepackage{multirow}
\usepackage[
    tocskip=0.1\baselineskip, skip=0.7\baselineskip, parfill
]{parskip}
\usepackage{listings}
\usepackage{fancyhdr}
\usepackage{xcolor}

% Kopf- und Fußzeile
\pagestyle{fancy}
\fancyhf{}
%Kopfzeile mittig mit Kaptilname
\fancyhead[C]{\nouppercase{\leftmark}}
%Fußzeile links bzw. innen
\fancyfoot[L]{\versuchsname}
%Fußzeile mittig (Seitennummer)
\fancyfoot[R]{\thepage}
\renewcommand{\footrulewidth}{0.35pt}

% Hilfs-Commands für Gleichungen
\newcommand{\widespace}{\enspace}
\newcommand{\wideeq}{\widespace = \widespace}
\newcommand{\wideneq}{\widespace \neq \widespace}
\newcommand{\wideapprox}{\widespace \approx \widespace}
\newcommand{\wideleq}{\widespace \leq \widespace}
\newcommand{\widegeq}{\widespace \geq \widespace}
\newcommand{\widele}{\widespace \le \widespace}
\newcommand{\widege}{\widespace \ge \widespace}
\newcommand{\wideiff}{\widespace \iff \widespace}
\newcommand{\wideimplies}{\widespace \implies \widespace}
\newcommand{\pd}[2]{
    \frac{\partial #1}{\partial #2}
}
\newcommand{\result}[2]{
    #1 \, \text{#2}
}

% Markieren von verwendetem Code
\definecolor{codebg}{RGB}{230, 240, 255}
\newcommand{\coderef}[1]{
    \text{
        \enspace
        \footnotesize
        \colorbox{codebg}{\texttt{#1()}}
    }
}

% Formatieren von Code im Anhang
\definecolor{codegreen}{rgb}{0,0.6,0}
\definecolor{codegray}{rgb}{0.5,0.5,0.5}
\definecolor{codepurple}{rgb}{0.58,0,0.82}
\definecolor{backcolour}{rgb}{0.95,0.95,0.92}
\lstdefinestyle{mystyle}{
    backgroundcolor=\color{backcolour},   
    commentstyle=\color{codegreen},
    keywordstyle=\color{magenta},
    numberstyle=\tiny\color{codegray},
    stringstyle=\color{codepurple},
    basicstyle=\ttfamily\footnotesize,
    breakatwhitespace=false,         
    breaklines=true,                 
    captionpos=b,                    
    keepspaces=true,                 
    numbers=left,                    
    numbersep=5pt,                  
    showspaces=false,                
    showstringspaces=false,
    showtabs=false,                  
    tabsize=4
}
\lstset{style=mystyle}

% Formatierung von Absätzen
\renewcommand{\baselinestretch}{1.2}

\allowdisplaybreaks

% Allgemeine Infos
\newcommand{\versuchsname}{
    FHV - Franck-Hertz-Versuch
}
\newcommand{\githuburl}{
    \url{https://github.com/WeinSim/P3B/FHV}
}

% Titel und Autor
\title{\versuchsname}
\author{Simon Weinzierl, Yannic Werner}

\begin{document}

\maketitle

\begin{center}
    Physikalisches Fortgeschrittenenpraktikum P3B
    nach der Studienordnung für Studienbeginn bis WS 2022/23
\end{center}

\vspace*{6cm}

\begin{center}
    \footnotesize
    Alle Teile dieses Dokuments (Vorbereitung, Protokoll, Auswertung) wurden
    von beiden Teilnehmern in gleichen Teilen und ohne fremde Hilfe bearbeitet.
    Sofern fremde Quellen verwendet wurden, sind diese angegeben.

    Der \LaTeX-Code ist auf GitHub unter \githuburl verfügbar.
    
    © Alle Rechte vorbehalten.
\end{center}

 % LMU-Siegel
\AddToShipoutPicture*{
    \put(315,0){
        \parbox[b][5cm]{5cm}{
            \includegraphics[width=10cm]{Abbildungen/Siegel_LMU.pdf}
        }
    }
}

\newpage

% Inhaltsverzeichnis
\tableofcontents

\newpage

% Literatur
\bibliographystyle{alpha}
\bibliography{literatur}

\newpage

\section{Vobereitung}

\subsection{Physikalischer Hintergrund}

\subsubsection{Elektronenkonfiguration}

Elektronen werden in einem Atom (im Schrödinger-Bild ohne Spin) durch die drei
Quantenzahlen $n$, $l$ und $m$ beschrieben:
$n$ ist die Hauptquantenzahl und hat die größte Auswirkung auf
die Energie des Elektrons. $l$ wird als Drehimpulsquantenzahl bezeichnet und
gibt den Bahndrehimpuls an ($L^2 = l (l + 1) \hbar^2$).
$m$ ist die magnetische Quantenzahl und gibt die Komponente des Drehimpulses
entlang einer beliebigen Achse (i.d.R. der $z$-Achse) an.
Für die Wertebereiche von $l$ und $m$ gelten die Regeln
$0 \leq l < n$ und $-l \leq m \leq l$.
% Die wahrscheinlichen Aufenthaltsbereiche der Elektronen mit verschiedenen
% Quantenzahlen werden als Orbitale bezeichnet.

Das Pauli-Prinzip besagt, dass zwei Elektronen in einem Atom nie den gleichen
Satz an Quantenzahlen besitzen können. Hier muss jedoch der Spin mit berücksichtigt
werden. Weil die Spin-Quantenzahl $s$ für Elektronen stets die Werte
$1/2$ oder $-1/2$ annimmt, kann also jede Kombination von Quantenzahlen
$(n, l, m)$ von höchstens zwei Elektronen besetzt werden. Diese haben dann einen
entgegengesetzten Spin. In einem Atom mit Kernladungszahl $Z > 2$
befinden sich also selbst im Grundzustand des Atoms nur zwei Elektronen
im niedrigsten Energiezustand. Die übrigen $Z - 2$ Elektronen verteilen sich
auf höhere Energieniveaus, wobei niedrigere Energienievaus stets vor höhreren
aufgefüllt werden. Die genaue Besetzung dieser Energienviveaus bezeichnet
man als die Elektronenkonfiguration.
In der üblichen Schreibweise wird für jede Kombination an Haupt- und
Drehimpulsquantenzahl die Anzahl der Elektronen angegeben, die diese
Quantenzahlen besitzen. Die Drehimpulsquantenzahl $l$ wird dabei als Buchstabe
kodiert: $0 \to \text s, \, 1 \to \text p, \, 2 \to \text d, \, 3 \to \text f,
\, 4 \to \text g$. Die Hauptquantenzahl $n$ wird vor diesen Buchstaben geschrieben
und die Anzahl der Elektronen als Exponent. So werden alle Elektronen
in aufsteigender Energie aufgeschrieben.


\subsubsection{Termschema und Grotrian-Diagramm}

Für die Erklärung einiger quantenmechanischen Phänomene sind genauere Informationen
über die Elektronen in einem Atom notwendig als nur die Elektronenkonfiguration.
Insbesondere sind der Gesamtdrehimpuls $J$, welcher die Summe aus Bahn- und
Spindrehimpuls ist, und die sog. Multiplizität, die die Anzahl an möglichen
Spin-Projektionen auf die $z$-Achse angibt, relevant.
Die Multiplizität berechnet sich als $(2S + 1)$, wobei $S$ die Summe
der Spin-Porjektionen der einzelnen Elektronen im Atom ist.
Diese Informationen werden kompakt in der sog. Termschreibweise angegeben:
\[  
    {}^{2 S + 1} L_J.
\]
Zustände mit Multiplizität 1 werden als Singulett-Zustände bezeichnet,
Zustände mit Multiplizität 2 als Dublett, mit Multiplizität 3 als Triplett, etc.
Durch die Feinstruktur (s. \ref{LS-Kopplung}) werden Zustände mit gleichen Termen 
aufgespalten; die Anzahl an entstehenden Linien ist durch die Multiplizität gegeben.
Die Multiplizität ist außerdem deswegen relevant, weil nach den Hund'schen
Regeln Terme mit höherer Multiplizität eine geringere Energie besitzen.
Dadurch wird die Reihenfolge bestimmt, mit der einzelne Subschalen aufgefüllt werden:
Zuerst wird jeder Wert von $m$ von einem Elektron besetzt.
Wenn alle $2l + 1$ Drehimpulsprojektionen besetzt sind (und damit $2l + 1$
ungepaarte Elektronen mit gleichgeritetem Spin vorliegen) werden die
magnetischen Quantenzahlen doppelt besetzt (von Elektronen mit entgegengesetztem
Spin, um das Pauli-Prinzip nicht zu verletzen).

Vollständig gefüllte Subschalen müssen für die Bestimmung eines Terms nicht
berücksichtigt werden, weil sie nichts zum Gesamtdrehimpuls und -Spin beitragen.
Für jedes Elektron mit Quantenzahlen $(n, l, m)$ gibt es ein Elektron mit
Quantenzahlen $(n, l, -m)$ und je zwei Elektronen mit den selben Quantenzahlen
haben entgegengesetzten Spin.

Verschiedene Zustände in einem Atom werden also von verschiedenen Termen beschrieben.
Somit kann jedem Term ein Energieniveau zugeordnet werden. Diese Energieniveaus
können in einem Grotrian-Diagramm (s. \autoref{Grotrian_Helium}) aufgetragen werden.
Dabei werden Zustände mit unterschiedlicher Multiplizität
(z.B. Singulett- und Triplett-Zustände) meist voneinander getrennt aufgetragen,
weil insbesondere bei leichten Atomen Übergänge zwischen Zuständen unterschiedlicher 
Multiplizität verboten sind. Jeder Übergang entspricht einer charakteristischen
Energie (und damit einer charakteristischen beobachtbaren Wellenlänge),
die mithilfe von Spektroskopie gemessen werden können, wodurch Rückschlüsse
über die Eigenschaften der untersuchten Probe gezogen werden können.

\begin{figure}[H]
    \centering
    \includegraphics[width=0.4\linewidth]{Abbildungen/Termschema_Demtröder.png}
    \caption{Grotrian-Diagramm für Helium. Aus \cite[183]{demtröder}}
    \label{Grotrian_Helium}
\end{figure}


\subsubsection{LS-Kopplung}
\label{LS-Kopplung}

Der Spin $\mathbf s$ (welcher in der Schrödingergleichung nicht auftritt, sondern
erst durch Berücksichtigung relativistischer Effekte aus der Dirac-Theorie stammt)
verleiht den Elektronen ein kleines intrinsisches magnetisches Dipolmoment
$\boldsymbol \mu_S = g_s (\mu_B / \hbar) \mathbf s$. Dabei hat $g_s$ einen Wert
von $\approx 2$ und wird Landé-Faktor genannt.
Dieses magnetische Moment wechselwirkt mit dem magnetischen Moment des Atomkerns,
welches aus Sicht des Elektrons durch dessen Bahndrehimpuls erzeugt wird.
Diese Interaktion wird als LS-Kopplung bezeichnet.
Es ergibt sich eine Energiekorrektur der Terme, die proportional zum Skalarprodukt
$\mathbf L \cdot \mathbf s$ des Bahn- und Spindrehimpuls ist.
Weil in einem Mehrelektronensystem mit Gesamtspin $S$ die Projektion des Spins
auf die $z$-Achse (und damit auch das Skalaprodukt
$\mathbf L \cdot \mathbf S$) genau $2S + 1$ verschiedene Werte annehmen kann,
spaltet sich das Energieniveau eines Terms mit Multiplizität $M$ in $M$ verschiedene
voneinander getrennte Niveaus auf.
Terme mit gleichgerichtetem Bahndrehimpuls und Spin besitzen dabei die
niedrigste Energie.
Diese Aufspaltung wird als Feinstruktur bezeichnet.

\cite[155--161]{demtröder}


\subsubsection{Stoßionisation}

Stößt ein Elektron mit genügend kinetischer Energie auf ein (neutrales) Atom $A$,
so kann es passieren, dass aus $A$ ein weiteres Elektron herausgeschlagen wird.
Nach dem Stoßprozess liegt das positiv geladene Ion $A^+$ vor, sowie zwei
freie Elektronen. Dieser Prozess kann als Reaktionsgleichung geschrieben werden:
\[
    A + e^- \quad \longrightarrow \quad A^+ + 2 e^-.
\]
Für die Energiebilanz dieser Reaktion muss gelten
\[
    E_\text{kin} + E_B \wideeq E_1 + E_2,
\]
wobei $E_\text{kin}$ die kinetische Energie des Elektrons vor dem Stoß ist,
$E_{1/2}$ die kinetischen Energien der beiden Elektronen nach dem Stoß
und $E_B$ die Bindungsenergie des ursprünglich gebundenen Elektrons im Atom.

Die Wahrscheinlichkeit, dass ein neutrales Atom bei einem Stoß mit einem Elektron
tatsächlich ionisiert wird, hängt von der Atomsorte, der kinetischen Energie des
freien Elektrons und der Bindungsenergie des gebundenen Elektrons ab.
Sie wird mithilfe des Ionisierungsquerschnitts $\sigma$ beschrieben.
$\sigma$ ist abhängig von der kinetischen Energie des freien Elektrons und gibt
die Querschnittsfläche um das Atom $A$ an, durch die das freie Elektron
fliegen muss.

\cite[34--35]{demtröder}


\subsubsection{Unselbstständige Dunkel- bzw. Townsend-Entladung}

Die Townsend-Entladung ist eine Art der Gasentladung, bei der sich ein geladener
Plattenkondensator über ein Gas, welches sich zwischen dessen Platten befindet,
entladen wird. Im Regime der unselbstständigen Townsend-Entladung
wird eine externe Elektronenquelle benötigt (etwa eine UV-Lampe, die durch den
Photoelektrischen Effekt Elektronen an der Kathode freisetzt).
Die Elektronen werden im elektrischen Feld des Kondensator richtung Anode
beschleunigt. Dabei können sie auf neutrale Gasteilchen treffen und diese
ionisieren, wodurch weitere Elektronen freigesetzt werden.
Dadurch entsteht ein Lawinen-Effekt, weil entlang des Kondensators immer mehr
Elektronen freigesetzt werden.
Die dabei entstehenden positiv geladenen Ionen werden zur Kathode hin beschleunigt
und können beim Auftreffen auf dieser weitere Elektronen freisetzen.
Wenn auf diese Art und Weise genügend freie Elektronen an der Kathode entstehen,
um den Prozess von alleine aufrecht zu erhalten, spricht man von der selbstständigen
Townsend-Entladung.

\cite[18--21]{koubek}


\subsubsection{Erklärung des 1. und 2. Townsend-Koeffizienten}

Der Strom, der bei der Townsend-Entladung zwischen den Platten des Kondensators
fließt, kann mit der Formel
\[
    \frac{I}{I_0} \wideeq
    \frac{
        e^{\alpha d}
    }{
        1 - \lambda (e^{\alpha d} - 1)
    }
    % Quelle?
\]
berechnet werden.

\cite[18--21]{koubek}


\subsection{Aufgabe zur Vorbereitung}

\newpage

\section{Veruschsablaufplan}

\subsection{Benötigte Materialien}
    \begin{enumerate}[label=\arabic*.]
        \item Eins
        \item Zwei
    \end{enumerate}

\newpage

\subsection{Teilversuch 1: Bragg-Reflexion von Röntgenstrahlung des Molybdän an einem NaCl-Eiskristall}
\begin{enumerate}[label = (\Roman*)]
    \item Ziel: ...
    
    \item Versuchsmethode: ...
    
    \item Versuchsskizze:
    
        \begin{figure}[H]
        \centering
        \includegraphics[width=0.7\linewidth]{Bild}
        \caption{Versuchsskizze Teilversuch 1}
        \end{figure}

    \item Planung der Durchführung
        \begin{itemize}
            \item eins
            \item zwei
        \end{itemize}

    \item Vorüberlegungen zur Durchführung \& Auswertung
        \begin{itemize}
            \item eins
            \item zwei
        \end{itemize}
    
\end{enumerate}

\newpage

\subsection{Teilversuch 2: Energiespektrum einer Röntgenröhre in Abhängigkeit der Spannung}
\begin{enumerate}[label = (\Roman*)]
    \item Ziel: 
    
    \item Versuchsmethode: 
    
    \item Versuchsskizze:
    
        \begin{figure}[H]
        \centering
        \includegraphics[width=0.7\linewidth]{Bild}
        \caption{Versuchsskizze Teilversuch 2}
        \end{figure}

    \item Planung der Durchführung
        \begin{itemize}
           \item eins
           \item zwei
        \end{itemize}

    \item Vorüberlegungen zur Durchführung \& Auswertung
        \begin{itemize}
            \item eins
            \item zwei
        \end{itemize}
        
\end{enumerate}


\newpage

\section{Versuchsprotokoll}

Auf den folgenden Seiten befindet sich das eingescannte Versuchsprotokoll.
Alle Daten wurden selbst gemessen. Sofern fremde Hilfe benutzt wurde,
wurde sie klar gekennzeichnet.

Messunsicherheiten wurden angegeben und folgend in der Auswertung verwendet.
Alle weiteren Rechnungen und Analysen finden in der Versuchsasuwertung statt.

\includepdf[pages={...}, pagecommand={\thispagestyle{scrheadings}}, frame=true]{[Name.pdf]]}

\newpage

\section{Auswertung}

    \subsection{Teilversuch 1: Bragg-Reflexion von Röntgenstrahlung des Molybdän an einem NaCl-Eiskristall}

    \newpage

\section{Anmerkung: Graphische Auswertung und Fehlerfortpflanzung mit Python-Code}

Alle Berechnungen inkl. Fehlerbestimmung wurden mit einem selbstgeschriebenen
Python-Skript durchgeführt, um uns die Arbeit zu erleichtern und Fehler zu
vermeiden. Alle Ergebnisse, die auf diese Weise zustande gekommen sind,
sind entsprechend mit einem \colorbox{codebg}{blauen Hintergrund} gekennzeichnet;
s. folgendes Beispiel:
\[
    F \wideeq ma \wideeq \result{20}{kg} \cdot 9,81 \, \frac{\text m}{\text s^2}
    \wideeq \result{(19,62 \pm 0,5)}{N} \coderef{tv1}
\]
Dies soll bedeuten, dass die Berechnung des Wertes und der Unsicherheit von der
Python-Funktion namens \verb|tv1| durchgeführt wird.
Die Unsicherheit wird mithilfe der Gauß'schen Fehlerfortpflanzung berechnet.
Außerdem wird das Python-Package \texttt{matplotlib} zum Erstellen
von Graphen verwendet.

% Außerdem verwenden wir Python (in Kombination mit \texttt{matplotlib})
% zum Erstellen von Graphen. \texttt{numpy} wird (unter anderem) zum Bestimmen der
% Ausgleichsgeraden verwendet. Zum Bestimmen des Fehlerstreifens wird eine
% selbstgeschriebene Funktion verwendet, die das im AMW-Skript veschriebene Verfahren
% implementiert (2 äquidistante, zur Ürsprünglichen parallelen Geraden finden,
% zwischen denen 2/3 der Messpunkte liegen. Unsicherheit der Steigung:
% halbe Differenz der Steigung der "Diagonalen" durch den Fehlerstreifen).
% Graphen sind zusätzlich zum Namen der Funktion, in der sie erstellt werden,
% mit dem Dateinamen versehen, unter dem sie abgespeichert werden.

% Die Funktionen zu den einzelnen Teilversuchen befinden sich in der Datei
% \verb|Main.py|. Der Code zum Berechnen von bestimmten Ausdrücken
% inkl. deren Unsicherheit befindet sich in der Datei \verb|Expressions.py|.
% Das Output des Codes befindet sich in der Datei \verb|output.txt|.

Der verwendete Code ist sowohl auf GitHub verfügbar (\githuburl) als auch auf den
folgenden Seiten zu finden und kann mit dem Befehl \texttt{python Main.py}
ausgeführt werden. Für eine genauere Beschreibung des Codes siehe die README-Datei
auf GitHub sowie die Kommentare im Code.
(Manche Sonderzeichen im Code (ä, ö, ü, $\Delta$, etc.) werden von \LaTeX nicht
richtig erkannt, deswegen kann der Code auf den nachfolgenden Seiten an einigen
Stellen unvollständig erscheinen. Auf GitHub wird aber alles richtig angezeigt.)

\newpage


\verb|Main.py|:
\lstinputlisting[language=Python]{Code/Main.py}
\newpage

\verb|Expressions.py|:
\lstinputlisting[language=Python]{Code/Expressions.py}
\newpage

Output:
\lstinputlisting{Code/Output.txt}

\end{document}